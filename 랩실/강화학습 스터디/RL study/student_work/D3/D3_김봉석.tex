% Options for packages loaded elsewhere
\PassOptionsToPackage{unicode}{hyperref}
\PassOptionsToPackage{hyphens}{url}
%
\documentclass[
  a4paper]{article}
\usepackage{lmodern}
\usepackage{amssymb,amsmath}
\usepackage{ifxetex,ifluatex}
\ifnum 0\ifxetex 1\fi\ifluatex 1\fi=0 % if pdftex
  \usepackage[T1]{fontenc}
  \usepackage[utf8]{inputenc}
  \usepackage{textcomp} % provide euro and other symbols
\else % if luatex or xetex
  \usepackage{unicode-math}
  \defaultfontfeatures{Scale=MatchLowercase}
  \defaultfontfeatures[\rmfamily]{Ligatures=TeX,Scale=1}
  \setmonofont[]{Consolas}
\fi
% Use upquote if available, for straight quotes in verbatim environments
\IfFileExists{upquote.sty}{\usepackage{upquote}}{}
\IfFileExists{microtype.sty}{% use microtype if available
  \usepackage[]{microtype}
  \UseMicrotypeSet[protrusion]{basicmath} % disable protrusion for tt fonts
}{}
\makeatletter
\@ifundefined{KOMAClassName}{% if non-KOMA class
  \IfFileExists{parskip.sty}{%
    \usepackage{parskip}
  }{% else
    \setlength{\parindent}{0pt}
    \setlength{\parskip}{6pt plus 2pt minus 1pt}}
}{% if KOMA class
  \KOMAoptions{parskip=half}}
\makeatother
\usepackage{xcolor}
\IfFileExists{xurl.sty}{\usepackage{xurl}}{} % add URL line breaks if available
\IfFileExists{bookmark.sty}{\usepackage{bookmark}}{\usepackage{hyperref}}
\hypersetup{
  pdftitle={Lecture D3},
  pdfauthor={Bong Seok Kim},
  hidelinks,
  pdfcreator={LaTeX via pandoc}}
\urlstyle{same} % disable monospaced font for URLs
\usepackage[margin=1in]{geometry}
\usepackage{color}
\usepackage{fancyvrb}
\newcommand{\VerbBar}{|}
\newcommand{\VERB}{\Verb[commandchars=\\\{\}]}
\DefineVerbatimEnvironment{Highlighting}{Verbatim}{commandchars=\\\{\}}
% Add ',fontsize=\small' for more characters per line
\newenvironment{Shaded}{}{}
\newcommand{\AlertTok}[1]{\textcolor[rgb]{1.00,0.00,0.00}{#1}}
\newcommand{\AnnotationTok}[1]{\textcolor[rgb]{0.00,0.50,0.00}{#1}}
\newcommand{\AttributeTok}[1]{#1}
\newcommand{\BaseNTok}[1]{#1}
\newcommand{\BuiltInTok}[1]{#1}
\newcommand{\CharTok}[1]{\textcolor[rgb]{0.00,0.50,0.50}{#1}}
\newcommand{\CommentTok}[1]{\textcolor[rgb]{0.00,0.50,0.00}{#1}}
\newcommand{\CommentVarTok}[1]{\textcolor[rgb]{0.00,0.50,0.00}{#1}}
\newcommand{\ConstantTok}[1]{#1}
\newcommand{\ControlFlowTok}[1]{\textcolor[rgb]{0.00,0.00,1.00}{#1}}
\newcommand{\DataTypeTok}[1]{#1}
\newcommand{\DecValTok}[1]{#1}
\newcommand{\DocumentationTok}[1]{\textcolor[rgb]{0.00,0.50,0.00}{#1}}
\newcommand{\ErrorTok}[1]{\textcolor[rgb]{1.00,0.00,0.00}{\textbf{#1}}}
\newcommand{\ExtensionTok}[1]{#1}
\newcommand{\FloatTok}[1]{#1}
\newcommand{\FunctionTok}[1]{#1}
\newcommand{\ImportTok}[1]{#1}
\newcommand{\InformationTok}[1]{\textcolor[rgb]{0.00,0.50,0.00}{#1}}
\newcommand{\KeywordTok}[1]{\textcolor[rgb]{0.00,0.00,1.00}{#1}}
\newcommand{\NormalTok}[1]{#1}
\newcommand{\OperatorTok}[1]{#1}
\newcommand{\OtherTok}[1]{\textcolor[rgb]{1.00,0.25,0.00}{#1}}
\newcommand{\PreprocessorTok}[1]{\textcolor[rgb]{1.00,0.25,0.00}{#1}}
\newcommand{\RegionMarkerTok}[1]{#1}
\newcommand{\SpecialCharTok}[1]{\textcolor[rgb]{0.00,0.50,0.50}{#1}}
\newcommand{\SpecialStringTok}[1]{\textcolor[rgb]{0.00,0.50,0.50}{#1}}
\newcommand{\StringTok}[1]{\textcolor[rgb]{0.00,0.50,0.50}{#1}}
\newcommand{\VariableTok}[1]{#1}
\newcommand{\VerbatimStringTok}[1]{\textcolor[rgb]{0.00,0.50,0.50}{#1}}
\newcommand{\WarningTok}[1]{\textcolor[rgb]{0.00,0.50,0.00}{\textbf{#1}}}
\usepackage{graphicx,grffile}
\makeatletter
\def\maxwidth{\ifdim\Gin@nat@width>\linewidth\linewidth\else\Gin@nat@width\fi}
\def\maxheight{\ifdim\Gin@nat@height>\textheight\textheight\else\Gin@nat@height\fi}
\makeatother
% Scale images if necessary, so that they will not overflow the page
% margins by default, and it is still possible to overwrite the defaults
% using explicit options in \includegraphics[width, height, ...]{}
\setkeys{Gin}{width=\maxwidth,height=\maxheight,keepaspectratio}
% Set default figure placement to htbp
\makeatletter
\def\fps@figure{htbp}
\makeatother
\setlength{\emergencystretch}{3em} % prevent overfull lines
\providecommand{\tightlist}{%
  \setlength{\itemsep}{0pt}\setlength{\parskip}{0pt}}
\setcounter{secnumdepth}{-\maxdimen} % remove section numbering
% pdf

\usepackage[style=british]{csquotes}

% Hangul in rchunk 2019-02-12  
% Not the output yet
% Finally for italic font! (2020-04-28)
\usepackage{fontspec}
\usepackage[hangul]{kotex}
\setmainfont[]{Palatino Linotype}
\setmainhangulfont{NanumMyeongjo}[AutoFakeSlant]


\usepackage{pdflscape}
\usepackage{colortbl}
% \usepackage[table]{xcolor}
\newcolumntype{U}{>{\columncolor[gray]{0.8}}c}
\usepackage{tabularx,booktabs}
\usepackage{boxedminipage}
\usepackage{graphicx}
\usepackage{rotating}
\usepackage{rotfloat}
\usepackage{booktabs}
\usepackage{longtable}
\usepackage{subfigure}
\usepackage{wrapfig}
\usepackage{multirow}
\usepackage{titlepic}
\usepackage{graphicx}

% Shading code blocks (2020-03-27)
% Try other colors!
\usepackage{xcolor}
\usepackage{amsthm}
\usepackage{framed}
\renewenvironment{Shaded}{\colorlet{shadecolor}{blue!3}\begin{shaded}}{\end{shaded}}

% Quotation environment (2020-06-12)
\usepackage{quoting,xparse}
% \NewDocumentEnvironment{pquotation}
  % {\begin{quoting}[
  %    indentfirst=true,
  %    leftmargin=\parindent,
  %    rightmargin=\parindent]\itshape}
  % {\end{quoting}}
% \renewenvironment{quote}{\begin{pquotation}}{\end{pquotation}}
\renewenvironment{quote}{\colorlet{shadecolor}{green!3}\begin{shaded}\begin{quoting}[
  indentfirst=true,
  leftmargin=0.5in,
  rightmargin=0.5in]\itshape}
{\end{quoting}\end{shaded}}

% vspace shortcut (2018-05-23)
\def\br{\vspace{1pt}} % line break

% split vertically (2018-12-27)
\def\lc48{\begin{minipage}{0.48\textwidth}}
\def\rc48{\end{minipage}\begin{minipage}{0.48\textwidth}}
\def\ec48{\end{minipage}}

\def\lc24{\begin{minipage}{0.24\textwidth}}
\def\rc24{\end{minipage}\begin{minipage}{0.24\textwidth}}
\def\ec24{\end{minipage}}

% \setfont{quote}{shape=\itshape,family=\rmfamily}

\title{Lecture D3}
\author{Bong Seok Kim}
\date{2021-01-20}

\begin{document}
\maketitle

{
\setcounter{tocdepth}{2}
\tableofcontents
}
\hypertarget{exercise-1}{%
\subsection{Exercise 1}\label{exercise-1}}

How would you genalize this game with arbitraty value of \(m_1\)
(minimum increment),\(m_2\) (maximum increment), and \(N\) (the winning
number)?

\hypertarget{solution}{%
\subsubsection{solution:}\label{solution}}

\(m_1=1\), \(m_2=2\), \(N=31\), you should always do action to go
\(S_{28},S_{25},\dots\)

\(m_1=2\), \(m_2=5\), \(N=50\), you should always do action to go
\(S_{43}\dots\)

\(\dots\)

in general

you should do action to go \(S_{N-k(m1+m2)}\)

then, \(\pi{(s)}^*=N-k(m1+m2)-s\), where \(N-k(m1+m2)-s \in[m1,m2]\)

\newpage

\hypertarget{exercise-3}{%
\subsection{Exercise 3}\label{exercise-3}}

There is only finite number of \(deterministic\) \(stationary\) policy.
How many is it?

\hypertarget{solution-1}{%
\subsubsection{solution :}\label{solution-1}}

\begin{enumerate}
  \item $deterministic$ policy gives an single action for each state.
  \item $stationary$ is, a policy that does not change over time
  \item $deterministic$ $stationary$ policy deterministically selects actions based on the current state with mapping with no loss of generality
\end{enumerate}

\(\therefore\) a number of deterministic stationary policy is
\(|A|^{|S|}\)

\newpage

\hypertarget{exercise-4}{%
\subsection{Exercise 4}\label{exercise-4}}

Formulate the first example in this lecture note using the terminology
including state,action, reward, policy, transition. Describe the optimal
policy using the terminology as well.

\textbf{State} :

\(S = \{1,2,\dots ,31\}\)

\textbf{Action} :

\(A = \{a_1,a_2\}\)

\textbf{Reward} :

\(R(30,a_1)=R(29,a_2)=1\) all ohter \(R(s,a)\) = 0

\textbf{Transition} :

\(P_{ss^\prime}^a =P(S_{t+1}=S^{\prime} |S_t=s, A_t=a) =1\)

\(s^{\prime}=s+1, if(a=a1)\)

\(s^{\prime}=s+2, if(a=a2)\)

otherwise 0.

\textbf{Optimal Policy} :

\(\pi^*=argmax_{\pi}V_t(s)^\pi\)

S(3n-1) : a2

S(3n) : a1

\newpage

\hypertarget{exercise-5-uxd30cuxc774uxc36cuxc73cuxb85c-uxad6cuxd604-uxc5b4uxb5bbuxac8c-uxd560uxc9c0-uxc798-uxbaa8uxb974uxaca0uxc74c..}{%
\subsection{Exercise 5 (파이썬으로 구현 어떻게 할지 잘
모르겠음..)}\label{exercise-5-uxd30cuxc774uxc36cuxc73cuxb85c-uxad6cuxd604-uxc5b4uxb5bbuxac8c-uxd560uxc9c0-uxc798-uxbaa8uxb974uxaca0uxc74c..}}

\vspace{10pt}

\hypertarget{from-the-first-example}{%
\paragraph{From the first example,}\label{from-the-first-example}}

\begin{itemize}
\tightlist
\item
  Assume that your opponent increments by 1 with prob. 0.5 and by 2 with
  prob. 0.5.
\item
  Assume that the winning number is 10 instead of 31.
\item
  Your opponent played first and she called out 1.
\item
  Your current a policy \(\pi_{0}\) is that

  \begin{itemize}
  \tightlist
  \item
    If the current state s\textless=5 then increments by 2.
  \item
    If the current state s\textgreater5 then increments by 1.
  \end{itemize}
\end{itemize}

Evaluate \(V^{\pi_{0}}(1)\)

\hypertarget{uxc5b4uxb5bbuxac8c-value-iterationuxc73cuxb85c-p}{%
\subsubsection{어떻게 Value iteration으로 ?
P?}\label{uxc5b4uxb5bbuxac8c-value-iterationuxc73cuxb85c-p}}

\begin{Shaded}
\begin{Highlighting}[]
\ImportTok{import}\NormalTok{ numpy }\ImportTok{as}\NormalTok{ np}
\ImportTok{import}\NormalTok{ pandas }\ImportTok{as}\NormalTok{ pd}
\NormalTok{Winning_num }\OperatorTok{=} \DecValTok{10}

\NormalTok{States }\OperatorTok{=}\NormalTok{ np.arange(}\DecValTok{1}\NormalTok{,Winning_num}\OperatorTok{+}\DecValTok{1}\NormalTok{)}

\NormalTok{R }\OperatorTok{=}\NormalTok{ np.append(np.repeat(}\DecValTok{0}\NormalTok{,}\DecValTok{9}\NormalTok{),}\DecValTok{1}\NormalTok{).reshape(}\DecValTok{10}\NormalTok{,}\DecValTok{1}\NormalTok{)}

\NormalTok{P_my }\OperatorTok{=}\NormalTok{ np.array([ [ }\DecValTok{0}\NormalTok{, }\DecValTok{0}\NormalTok{, }\DecValTok{1}\NormalTok{, }\DecValTok{0}\NormalTok{, }\DecValTok{0}\NormalTok{, }\DecValTok{0}\NormalTok{, }\DecValTok{0}\NormalTok{, }\DecValTok{0}\NormalTok{, }\DecValTok{0}\NormalTok{, }\DecValTok{0}\NormalTok{ ],}
\NormalTok{                  [ }\DecValTok{0}\NormalTok{, }\DecValTok{0}\NormalTok{, }\DecValTok{0}\NormalTok{, }\DecValTok{1}\NormalTok{, }\DecValTok{0}\NormalTok{, }\DecValTok{0}\NormalTok{, }\DecValTok{0}\NormalTok{, }\DecValTok{0}\NormalTok{, }\DecValTok{0}\NormalTok{, }\DecValTok{0}\NormalTok{ ],}
\NormalTok{                  [ }\DecValTok{0}\NormalTok{, }\DecValTok{0}\NormalTok{, }\DecValTok{0}\NormalTok{, }\DecValTok{0}\NormalTok{, }\DecValTok{1}\NormalTok{, }\DecValTok{0}\NormalTok{, }\DecValTok{0}\NormalTok{, }\DecValTok{0}\NormalTok{, }\DecValTok{0}\NormalTok{, }\DecValTok{0}\NormalTok{ ],}
\NormalTok{                  [ }\DecValTok{0}\NormalTok{, }\DecValTok{0}\NormalTok{, }\DecValTok{0}\NormalTok{, }\DecValTok{0}\NormalTok{, }\DecValTok{0}\NormalTok{, }\DecValTok{1}\NormalTok{, }\DecValTok{0}\NormalTok{, }\DecValTok{0}\NormalTok{, }\DecValTok{0}\NormalTok{, }\DecValTok{0}\NormalTok{ ],}
\NormalTok{                  [ }\DecValTok{0}\NormalTok{, }\DecValTok{0}\NormalTok{, }\DecValTok{0}\NormalTok{, }\DecValTok{0}\NormalTok{, }\DecValTok{0}\NormalTok{, }\DecValTok{0}\NormalTok{, }\DecValTok{1}\NormalTok{, }\DecValTok{0}\NormalTok{, }\DecValTok{0}\NormalTok{, }\DecValTok{0}\NormalTok{ ],}
\NormalTok{                  [ }\DecValTok{0}\NormalTok{, }\DecValTok{0}\NormalTok{, }\DecValTok{0}\NormalTok{, }\DecValTok{0}\NormalTok{, }\DecValTok{0}\NormalTok{, }\DecValTok{0}\NormalTok{, }\DecValTok{1}\NormalTok{, }\DecValTok{0}\NormalTok{, }\DecValTok{0}\NormalTok{, }\DecValTok{0}\NormalTok{ ],}
\NormalTok{                  [ }\DecValTok{0}\NormalTok{, }\DecValTok{0}\NormalTok{, }\DecValTok{0}\NormalTok{, }\DecValTok{0}\NormalTok{, }\DecValTok{0}\NormalTok{, }\DecValTok{0}\NormalTok{, }\DecValTok{0}\NormalTok{, }\DecValTok{1}\NormalTok{, }\DecValTok{0}\NormalTok{, }\DecValTok{0}\NormalTok{ ],}
\NormalTok{                  [ }\DecValTok{0}\NormalTok{, }\DecValTok{0}\NormalTok{, }\DecValTok{0}\NormalTok{, }\DecValTok{0}\NormalTok{, }\DecValTok{0}\NormalTok{, }\DecValTok{0}\NormalTok{, }\DecValTok{0}\NormalTok{, }\DecValTok{0}\NormalTok{, }\DecValTok{1}\NormalTok{, }\DecValTok{0}\NormalTok{ ],}
\NormalTok{                  [ }\DecValTok{0}\NormalTok{, }\DecValTok{0}\NormalTok{, }\DecValTok{0}\NormalTok{, }\DecValTok{0}\NormalTok{, }\DecValTok{0}\NormalTok{, }\DecValTok{0}\NormalTok{, }\DecValTok{0}\NormalTok{, }\DecValTok{0}\NormalTok{, }\DecValTok{0}\NormalTok{, }\DecValTok{1}\NormalTok{ ],}
\NormalTok{                  [ }\DecValTok{0}\NormalTok{, }\DecValTok{0}\NormalTok{, }\DecValTok{0}\NormalTok{, }\DecValTok{0}\NormalTok{, }\DecValTok{0}\NormalTok{, }\DecValTok{0}\NormalTok{, }\DecValTok{0}\NormalTok{, }\DecValTok{0}\NormalTok{, }\DecValTok{0}\NormalTok{, }\DecValTok{1}\NormalTok{ ] ])}


\NormalTok{P_opp }\OperatorTok{=}\NormalTok{np.array([[}\DecValTok{0}\NormalTok{, }\FloatTok{0.5}\NormalTok{, }\FloatTok{0.5}\NormalTok{, }\DecValTok{0}\NormalTok{, }\DecValTok{0}\NormalTok{, }\DecValTok{0}\NormalTok{, }\DecValTok{0}\NormalTok{, }\DecValTok{0}\NormalTok{, }\DecValTok{0}\NormalTok{, }\DecValTok{0}\NormalTok{],}
\NormalTok{                 [}\DecValTok{0}\NormalTok{, }\DecValTok{0}\NormalTok{, }\FloatTok{0.5}\NormalTok{, }\FloatTok{0.5}\NormalTok{, }\DecValTok{0}\NormalTok{, }\DecValTok{0}\NormalTok{, }\DecValTok{0}\NormalTok{, }\DecValTok{0}\NormalTok{, }\DecValTok{0}\NormalTok{, }\DecValTok{0}\NormalTok{],}
\NormalTok{                 [}\DecValTok{0}\NormalTok{, }\DecValTok{0}\NormalTok{, }\DecValTok{0}\NormalTok{, }\FloatTok{0.5}\NormalTok{, }\FloatTok{0.5}\NormalTok{, }\DecValTok{0}\NormalTok{, }\DecValTok{0}\NormalTok{, }\DecValTok{0}\NormalTok{, }\DecValTok{0}\NormalTok{, }\DecValTok{0}\NormalTok{],}
\NormalTok{                 [}\DecValTok{0}\NormalTok{, }\DecValTok{0}\NormalTok{, }\DecValTok{0}\NormalTok{, }\FloatTok{0.5}\NormalTok{, }\FloatTok{0.5}\NormalTok{, }\DecValTok{0}\NormalTok{, }\DecValTok{0}\NormalTok{, }\DecValTok{0}\NormalTok{, }\DecValTok{0}\NormalTok{, }\DecValTok{0}\NormalTok{],}
\NormalTok{                 [}\DecValTok{0}\NormalTok{, }\DecValTok{0}\NormalTok{, }\DecValTok{0}\NormalTok{, }\DecValTok{0}\NormalTok{, }\FloatTok{0.5}\NormalTok{, }\FloatTok{0.5}\NormalTok{, }\DecValTok{0}\NormalTok{, }\DecValTok{0}\NormalTok{, }\DecValTok{0}\NormalTok{, }\DecValTok{0}\NormalTok{],}
\NormalTok{                 [}\DecValTok{0}\NormalTok{, }\DecValTok{0}\NormalTok{, }\DecValTok{0}\NormalTok{, }\DecValTok{0}\NormalTok{, }\DecValTok{0}\NormalTok{, }\DecValTok{0}\NormalTok{, }\FloatTok{0.5}\NormalTok{, }\FloatTok{0.5}\NormalTok{, }\DecValTok{0}\NormalTok{, }\DecValTok{0}\NormalTok{],}
\NormalTok{                 [}\DecValTok{0}\NormalTok{, }\DecValTok{0}\NormalTok{, }\DecValTok{0}\NormalTok{, }\DecValTok{0}\NormalTok{, }\DecValTok{0}\NormalTok{, }\DecValTok{0}\NormalTok{, }\DecValTok{0}\NormalTok{, }\FloatTok{0.5}\NormalTok{, }\FloatTok{0.5}\NormalTok{, }\DecValTok{0}\NormalTok{],}
\NormalTok{                 [}\DecValTok{0}\NormalTok{, }\DecValTok{0}\NormalTok{, }\DecValTok{0}\NormalTok{, }\DecValTok{0}\NormalTok{, }\DecValTok{0}\NormalTok{, }\DecValTok{0}\NormalTok{, }\DecValTok{0}\NormalTok{, }\DecValTok{0}\NormalTok{, }\FloatTok{0.5}\NormalTok{, }\FloatTok{0.5}\NormalTok{],}
\NormalTok{                 [}\DecValTok{0}\NormalTok{, }\DecValTok{0}\NormalTok{, }\DecValTok{0}\NormalTok{, }\DecValTok{0}\NormalTok{, }\DecValTok{0}\NormalTok{, }\DecValTok{0}\NormalTok{, }\DecValTok{0}\NormalTok{, }\DecValTok{0}\NormalTok{, }\DecValTok{0}\NormalTok{, }\DecValTok{1}\NormalTok{],}
\NormalTok{                 [}\DecValTok{0}\NormalTok{, }\DecValTok{0}\NormalTok{, }\DecValTok{0}\NormalTok{, }\DecValTok{0}\NormalTok{, }\DecValTok{0}\NormalTok{, }\DecValTok{0}\NormalTok{, }\DecValTok{0}\NormalTok{, }\DecValTok{0}\NormalTok{, }\DecValTok{0}\NormalTok{, }\DecValTok{1}\NormalTok{]])}

\NormalTok{P}\OperatorTok{=}\NormalTok{ (P_opp}\OperatorTok{+}\NormalTok{P_my)}\OperatorTok{/}\DecValTok{2}


\NormalTok{gamma}\OperatorTok{=}\FloatTok{0.9}

\NormalTok{epsilon }\OperatorTok{=}\DecValTok{10}\OperatorTok{**}\NormalTok{(}\OperatorTok{-}\DecValTok{8}\NormalTok{)}

\NormalTok{v_old }\OperatorTok{=}\NormalTok{ np.array(np.repeat(}\DecValTok{0}\NormalTok{,}\DecValTok{10}\NormalTok{)).reshape(}\DecValTok{10}\NormalTok{,}\DecValTok{1}\NormalTok{)}


\NormalTok{result }\OperatorTok{=}\NormalTok{ []}
\ControlFlowTok{while} \VariableTok{True}\NormalTok{:}
\NormalTok{    result.append(v_old.T)}
\NormalTok{    v_new }\OperatorTok{=}\NormalTok{ R}\OperatorTok{+}\NormalTok{gamma}\OperatorTok{*}\NormalTok{np.dot(P, v_old)}
    \ControlFlowTok{if}\NormalTok{ np.}\BuiltInTok{max}\NormalTok{(np.}\BuiltInTok{abs}\NormalTok{(v_new}\OperatorTok{-}\NormalTok{v_old)) }\OperatorTok{>}\NormalTok{ epsilon:}
\NormalTok{        v_old }\OperatorTok{=}\NormalTok{ v_new}
        \ControlFlowTok{continue}
    \ControlFlowTok{break}

\NormalTok{v_old}
\end{Highlighting}
\end{Shaded}

\begin{verbatim}
## array([[5.05401013],
##        [5.31270495],
##        [5.71652077],
##        [5.96516709],
##        [6.48053065],
##        [7.03307803],
##        [7.6443749 ],
##        [8.3249999 ],
##        [8.9999999 ],
##        [9.9999999 ]])
\end{verbatim}

\hypertarget{mc-uxc758uxbbf8uxac00-uxc5c6uxc74c}{%
\subsubsection{MC \ldots 의미가
없음}\label{mc-uxc758uxbbf8uxac00-uxc5c6uxc74c}}

\begin{Shaded}
\begin{Highlighting}[]

\CommentTok{###########}

\NormalTok{states }\OperatorTok{=}\BuiltInTok{list}\NormalTok{(}\BuiltInTok{range}\NormalTok{(}\DecValTok{1}\NormalTok{, }\DecValTok{11}\NormalTok{))}

\ImportTok{import}\NormalTok{ numpy }\ImportTok{as}\NormalTok{ np}

\KeywordTok{def}\NormalTok{ Environment(this_state):}
    \KeywordTok{global}\NormalTok{ turn}
    \ControlFlowTok{if}\NormalTok{ (turn }\OperatorTok{%} \DecValTok{2}\NormalTok{) }\OperatorTok{==} \DecValTok{0}\NormalTok{ :}
\NormalTok{        next_state }\OperatorTok{=}\NormalTok{ my_policy(this_state)}

    \ControlFlowTok{else}\NormalTok{:}
\NormalTok{        next_state }\OperatorTok{=}\NormalTok{ opp_policy(this_state)}

\NormalTok{    turn }\OperatorTok{+=} \DecValTok{1}

    \ControlFlowTok{return}\NormalTok{ next_state}



\KeywordTok{def}\NormalTok{ my_policy(this_state):}
    \ControlFlowTok{if}\NormalTok{ this_state }\OperatorTok{<=} \DecValTok{5}\NormalTok{:}
        \ControlFlowTok{return}\NormalTok{ this_state}\OperatorTok{+}\DecValTok{2}
    \ControlFlowTok{else}\NormalTok{ :}
        \ControlFlowTok{return}\NormalTok{ this_state}\OperatorTok{+}\DecValTok{1}


\KeywordTok{def}\NormalTok{ opp_policy(this_state):}
\NormalTok{    rand }\OperatorTok{=}\NormalTok{ np.random.uniform()}
    \ControlFlowTok{if}\NormalTok{ this_state }\OperatorTok{==} \DecValTok{1}\NormalTok{:}
        \ControlFlowTok{return}\NormalTok{ this_state}\OperatorTok{+}\DecValTok{1}

    \ControlFlowTok{elif}\NormalTok{ rand }\OperatorTok{<} \FloatTok{0.5}\NormalTok{:}
         \ControlFlowTok{return}\NormalTok{ this_state}\OperatorTok{+}\DecValTok{1}

    \ControlFlowTok{else}\NormalTok{ :}
        \ControlFlowTok{return}\NormalTok{ this_state}\OperatorTok{+}\DecValTok{2}


\CommentTok{## Simulation}
\NormalTok{win }\OperatorTok{=} \DecValTok{0}

\CommentTok{#for i in range(1000):}
 \CommentTok{#   turn = 1}
  \CommentTok{#  done_mark = 1}
   \CommentTok{# this_state = 1}
   \CommentTok{# while done_mark:}
    \CommentTok{#    next_state = Environment(this_state)}

\CommentTok{#        this_state = next_state}

        \CommentTok{#print(this_state)}

 
 \CommentTok{#       if (turn %2 == 0) and this_state == 10:}
  \CommentTok{#          win += 1}
   \CommentTok{#         print("win game_over")}
    \CommentTok{#        done_mark=0}

     \CommentTok{#   else:}
      \CommentTok{#      print("lose game_over")}
      \CommentTok{#      done_mark=0}
\end{Highlighting}
\end{Shaded}

\begin{Shaded}
\begin{Highlighting}[]
\StringTok{"Done, Lecture D3 "}
\end{Highlighting}
\end{Shaded}

\begin{verbatim}
## [1] "Done, Lecture D3 "
\end{verbatim}

\end{document}
